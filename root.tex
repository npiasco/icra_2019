%%%%%%%%%%%%%%%%%%%%%%%%%%%%%%%%%%%%%%%%%%%%%%%%%%%%%%%%%%%%%%%%%%%%%%%%%%%%%%%%
%2345678901234567890123456789012345678901234567890123456789012345678901234567890
%        1         2         3         4         5         6         7         8

\documentclass[letterpaper, 10 pt, conference]{ieeeconf}  % Comment this line out if you need a4paper

%\documentclass[a4paper, 10pt, conference]{ieeeconf}      % Use this line for a4 paper

\IEEEoverridecommandlockouts                              % This command is only needed if 
                                                          % you want to use the \thanks command

%In case you encounter the following error:
%Error 1010 The PDF file may be corrupt (unable to open PDF file) OR
%Error 1000 An error occurred while parsing a contents stream. Unable to analyze the PDF file.
%This is a known problem with pdfLaTeX conversion filter. The file cannot be opened with acrobat reader
%Please use one of the alternatives below to circumvent this error by uncommenting one or the other
%\pdfobjcompresslevel=0
%\pdfminorversion=4

% See the \addtolength command later in the file to balance the column lengths
% on the last page of the document

% The following packages can be found on http:\\www.ctan.org
\usepackage{graphicx} % for pdf, bitmapped graphics files
%\usepackage{epsfig} % for postscript graphics files
%\usepackage{mathptmx} % assumes new font selection scheme installed
%\usepackage{times} % assumes new font selection scheme installed
\usepackage{amsmath} % assumes amsmath package installed
\usepackage{amssymb}  % assumes amsmath package installed
\usepackage{tabularx}
\usepackage{multirow}
\usepackage{tabularx}
\usepackage{xcolor}

\title{\LARGE \bf
Learning Scene Geometry for Visual\\Localization in Challenging Conditions
}

%\author{Nathan Piasco, D\'esir\'e Sidib\'e, Val\'erie Gouet-Brunet and C\'edric Demonceaux}% <-this % stops a space

\author{Nathan Piasco${}^{1,2}$, D\'esir\'e Sidib\'e${}^1$, Val\'erie Gouet-Brunet${}^2$ and C\'edric Demonceaux${}^1$% <-this % stops a space
\thanks{${}^1$ Le2i, ERL CNRS VIBOT 6000,  Universit\'e  Bourgogne Franche-Comt\'e}% <-this % stops a space
\thanks{${}^2$ Univ. Paris-Est, LaSTIG MATIS, IGN, ENSG, F-94160 Saint-Mand\'e, France}%
}

\begin{document}

\newcommand{\norm}[1]{\left\lVert#1\right\rVert}
\newcommand\numberthis{\addtocounter{equation}{1}\tag{\theequation}}

\maketitle
\thispagestyle{empty}
\pagestyle{empty}


%%%%%%%%%%%%%%%%%%%%%%%%%%%%%%%%%%%%%%%%%%%%%%%%%%%%%%%%%%%%%%%%%%%%%%%%%%%%%%%%
\begin{abstract}
We propose a new approach for outdoor large scale image based localization that can deal with challenging scenarios like cross-season, cross-weather, day/night and long-term localization. The key component of our method is a new learned global image descriptor, that can effectively benefit from scene geometry information during training. At test time, our system is capable of inferring the depth map related to the query image and use it to increase localization accuracy.

We are able to increase recall@1 performances by 2.15\% on cross-weather and long-term localization scenario and by 4.24\% points on a challenging winter/summer localization sequence versus state-of-the-art methods. Our method can also use weakly annotated data to localize night images across a reference dataset of daytime images.
\end{abstract}

\section{Introduction}
\label{sec:intro}

Visual-Based Localization (VBL) is a central topic in robotics and computer vision applications~\cite{Piasco2017}. It consists in retrieving the location of a visual query according to a known absolute reference. VBL is used in many applications such as autonomous driving, augmented reality, urban navigation or automatic map updating. In this paper, we address the VBL as an image retrieval problem where an input image is compared to a reference pool of localized images. This image-retrieval-like problem is two-step: descriptor computation for both the query and the reference images and similarity association across the descriptors. Since the reference images are associated to a location, by ranking images according to their similarity scores we obtain a approximate location for the query. Numerous works have introduced image descriptors well suited for image retrieval for localization~\cite{Arandjelovic2017,Kim2017a,Gordo2017,Radenovic2017,Liu2018}. 

One of the main challenges of image-based localization remains the mapping of images acquired under changing conditions: cross-season images matching~\cite{Naseer2017a}, long-term localization~\cite{Toft2018}, day to night place recognition~\cite{Torii2015}, etc. Recent approaches use complementary information in order to address these visually challenging localization scenarios (geometric information through point cloud~\cite{Sattler2018,Schonberger2018} or depth maps~\cite{Christie2016}, semantic information~\cite{Ardeshir2014,Christie2016,Naseer2017a}). However geometric or semantic information are not always available, especially in robotic applications when the sensor or the computational load on the robot is limited.

In this paper, we propose a image descriptor that learns, from an image, the corresponding scene geometry, in order to deal with challenging outdoor large-scale image-based localization scenarios. We introduce geometric information during the training step to make our new descriptor robust to visual changes that occur between images taken at different times. Once trained, our system is used on images only to construct a powerful descriptor for image retrieval. This kind of system design is also known as side information learning~\cite{Hoffman2016}, as it uses geometric and radiometric information only during the training step and just radiometric data for the image localization. Our method is especially well suited for robot long-term localization when the perceptive sensor on the robot is limited to a camera~\cite{Middelberg2014}, while having access to the full scene geometry off-line~\cite{Paparoditis2012,Maddern2016,Wang2016}. 

The paper is organized as follows. We firstly revisit in section~\ref{sec:related_work} recent works related to our method, including:~state of the art image descriptors for large scale outdoor localization, method for localization in changing environment and side information learning approaches. In section~\ref{sec:method}, we describe in detail our new image descriptor trained with side depth information. We illustrate the effectiveness of the proposed method on four challenging scenarios in section~\ref{sec:experiments}. Section~\ref{sec:conclusion} finally concludes the paper.

\section{Related Work}
\label{sec:related_work}

\vspace{4pt}\noindent\textbf{Image descriptor for outdoor visual localization.} Standard image descriptors for image retrieval in the context of image localization are usually built by combining sparse features with an aggregation method, such as BoW or VLAD. Specific features re-weighting scheme dedicated to image localization have been introduced in~\cite{Arandjelovic2014}. Authors of~\cite{Sattler2016} introduce a re-ranking routine to improve the localization performances on large-scale outdoor area. More recently, \cite{Arandjelovic2017} introduces NetVLAD, a convolutional neural network that is trained to learn a well-suited image representation for image localization. Numerous other CNN image descriptors have been proposed in the literature~\cite{Kim2017a,Gordo2017,Radenovic2017,Sunderhauf2015a,Liu2018} and achieve state of the art results in image retrieval for localization. Therefore we use CNN image descriptors as base component in our system.

\vspace{4pt}\noindent\textbf{Localization in challenging condition.} In order to deal with visual changes in images taken at different times, \cite{Naseer2018} uses a combination of handcrafted and learned descriptors. \cite{Garg2018} introduces temporal consistency by using a sequence of images, while in our proposal we use only one image as input for our descriptor. In~\cite{Porav2018}, authors synthesize new images to match the appearance of reference images, for instance they synthesized daytime images from night images. Numerous works~\cite{Stenborg2018,Toft2018,Naseer2017a} enhance their visual descriptors by adding semantic information. Although semantic representation is robust for long term localization, it may be costly to obtain. Other methods rely on geometric information like point clouds~\cite{Sattler2018,Schonberger2018}, or 3D structures~\cite{Torii2015}. Geometric information has the advantage of remaining more stable across time comparing to visual information but is not always available. That is why we decide to use depth information as side information in combination with radiometric data to learn a powerful image descriptor.

\vspace{4pt}\noindent\textbf{Learning with side information.} Recent work from~\cite{Li2017b} casts the side information learning problem as a domain adaptation problem, where source domain includes multiples modalities and the target domain is composed of a single modality. Another successful method have been introduced in~\cite{Hoffman2016}: authors train a deep neural network to hallucinate features from a depth map only presented during the training process to improve objects detection in images. The closest work to ours, presented in~\cite{xu2017learning}, uses recreated thermal images to improve pedestrian detection on standard images only. Our system, inspired by~\cite{xu2017learning}, learns how to produce depth maps from images to enhance the description of these images.

\section{Method}
\label{sec:method}

\begin{figure}
	\center
	\includegraphics[width=\linewidth]{vect/our_method}
	\caption{\label{fig:our_method} \textbf{Image descriptors training with auxiliary depth data:} two encoders are used for extracting deep features map from the main image modality and the auxiliary reconstructed depth map (inferred from our deep decoder). These features are used to create intermediate descriptors that are finally concatenated in one final image descriptor.}
\end{figure}

\begin{figure}
	\center
	\includegraphics[width=\linewidth]{vect/hall_method}
	\caption{\label{fig:hall_method} \textbf{Hallucination network for image descriptors learning:} we train an hallucination network, inspired from~\cite{Hoffman2016}, for the task of global image description. Unlike the proposed method (see figure~\ref{fig:our_method}), hallucination network reproduces feature maps that would have been obtained by a network trained with depth map rather than the deep map itself.}
\end{figure}



\subsection{Overview}
\label{subsec:overview}
We design a new global image description for the task of image based localization. We first extract dense feature maps from an input image with a convolutional neural network encoder ($E_I$). This feature maps is subsequently used to build a compact representation of the scene ($d_I$). State of the art features aggregation methods can be used to construct the image descriptor, such as MAC~\cite{Radenovic2017} or NetVALD~\cite{Arandjelovic2017}. We enhance this standard image descriptor with side depth map information that is only available during the training process. To do so, a deep neural fully convolutional neural network decoder ($D_G$) is used to reconstruct the corresponding depth map according to the image. The reconstructed depth is then used to extract a global depth map descriptor. We follow the same procedure used for image: we extract deep feature maps with an encoder ($E_D$) before building the descriptor ($d_D$). Finally, the image descriptor and the depth map descriptor are concatenated into a single descriptor. Figure~\ref{fig:our_method} summarizes the whole process of our new method.

\subsection{Training routine}
\label{subsec:training}
Trainable parameters are $\theta_{I}$ the weights of encoder and associated descriptor $\{E_I, d_I\}$, $\theta_{D}$ the weights of the encoder and following descriptor $\{E_D, d_D\}$ and $\theta_{G}$ the weights of the decoder $D_G$ used for depth map generation. 

We follow standard procedure of descriptor learning methods for training our system based on triplet margin losses. A triplet $\{q_{im}, q_{im}^+, q_{im}^-\}$ is composed of an anchor image $q_{im}$, a positive example $q_{im}^+$ representing the same scene as the anchor and an unrelated negative example $q_{im}^-$.
The first triplet loss acting on $\{E_I, d_I\}$ is:
\begin{multline}
	\label{eq:triplet_loss}
	L_{\theta_{I}} = max\left(0, \lambda + \norm{f_{\theta_{I}}(q_{im}) - f_{\theta_{I}}(q_{im}^+)}_2  \right. \\	
	\left. - \norm{f_{\theta_{I}}(q_{im}) - f_{\theta_{I}}(q_{im}^-)}_2 \right),
\end{multline}
where $f_{\theta_{I}}(x_{im})$ is the global descriptor of image $x_{im}$. $\lambda$ is an hyperparameter controlling the margin between positive and negative examples. $f_{\theta_{I}}$ can be written as:
\begin{equation}
	\label{eq:desc_details}
	f_{\theta_{I}}(x_{im}) = d_I(E_I(x_{im})),
\end{equation}
where $E_I(x_{im})$ represents the deep feature maps extracted by the decoder and $d_I$ the function used to build the final descriptor from the feature.

We train the depth map encoder + descriptor $\{E_D, d_D\}$ in a same manner, equation~\ref{eq:triplet_loss} becoming:
\begin{multline}
	\label{eq:depth_triplet_loss}
	L_{\theta_{D}}  = max\left(0, \lambda + \norm{f_{\theta_{D}}(\hat{q}_{depth}) - f_{\theta_{D}}(\hat{q}_{depth}^+))}_2  \right. \\	
	\left. - \norm{f_{\theta_{D}}(\hat{q}_{depth}) - f_{\theta_{D}}(\hat{q}_{depth}^-)}_2 \right),
\end{multline}
where $f_{\theta_{D}}(x_{depth})$ is the global descriptor of depth map $x_{depth}$ and $\hat{x}_{depth}$ is the reconstructed depth map of image $x_{im}$ by the decoder $D_G$:
\begin{equation}
	\label{eq:generator}
	\hat{x}_{depth} = D_G(E_I(x_{im})).
\end{equation}
Decoder $D_G$ use the deep representation of image $x_{im}$ computed by encoder $E_I$ in order to reconstruct the scene geometry. Notice that even if the encoder $E_I$ is not especially trained for depth map reconstruction, its intern representation is rich enough to be used by the decoder $D_G$ for the task of depth map inference.

The final image descriptor is trained with the following loss:
\begin{multline}
	\label{eq:cat_triplet_loss}
	L_{\theta_{I},\theta_{D}} = max\left(0, \lambda + \norm{F_{\theta_{I},\theta_{D}}(q) - F_{\theta_{I},\theta_{D}}(q^+)}_2  \right. \\	
	\left. - \norm{F_{\theta_{I},\theta_{D}}(q) - F_{\theta_{I},\theta_{D}}(q^-)}_2 \right),
\end{multline}
where $F(x)$ denote the concatenation of image descriptor and depth map descriptor:
\begin{equation}
	\label{eq:cat_function}
	F_{\theta_{I},\theta_{D}}(x_{im}) = \left[ f_{\theta_{I}}(x_{im}), f_{\theta_{D}}(\hat{x}_{depth}) \right].
\end{equation}

In order to train the depth map generator, we use a simple $L_1$ loss function:
\begin{align*}
	L_{\theta_{G}} &= \norm{x_{depth} - \hat{x}_{depth}}_{1}\\
				&= \norm{x_{depth} - D_G(E_I(x_{im}))}_{1}. \numberthis \label{eq:l1_loss}
\end{align*}

The whole system is trained according to the following constraints:
\begin{align}
	\left( \theta_{I}, \theta_{D} \right) & := arg\,\underset{\theta_{I}, \theta_{D}}{min} \left[ L_{\theta_{I}} + L_{\theta_{D}} + L_{\theta_{I},\theta_{D}} \right], \label{eq:sys_optimization_1} \\ 	
	\left( \theta_{G} \right) & := arg\,\underset{\theta_{G}}{min} \left[ L_{\theta_{G}} \right]. 	\label{eq:sys_optimization_2}
\end{align}

In order to train our system, we use two different optimizers: one updating $\theta_{I}$ and $\theta_{D}$ weights regarding constraint~\ref{eq:sys_optimization_1} and the other updating $\theta_{G}$ weights regarding constraint~\ref{eq:sys_optimization_2}. Because decoder $D_G$ is using feature maps computed by encoder $E_I$ (see equation~\ref{eq:generator}), at each optimization step on $\theta_{I}$ we need to update decoder weights $\theta_{G}$ to take in account possible changes in the image features. We finally train our entire system alternating between the optimization of weights $\{\theta_{I}, \theta_{D}\}$ and $\{\theta_G\}$ until convergence.

%\subsection{Fine tuning with weakly annotated data}
%\label{subsec:data}
%As we use triplet margin losses for training encoders weights $(\theta_{I}, \theta_{D})$, we need to gather images triplets. Triplets can be constructed thanks to an absolute position information about images, like GPS-tag~\cite{Arandjelovic2017,Liu2018} or by autonomous computer vision algorithm like Structure from Motion (SfM)~\cite{Godard2017,Radenovic2017,Kim2017a}. In addition, images used for triplets creation must represent the same scenes over different period of time to make the image descriptor robust to visual changes.
%
%In contrast the decoder part of our system $(\theta_{G})$ only needs image depth map aligned pair to be trained. Such data can be easily collected by a mapping mobile system, without using any post-processing. That means we have much more data available for training the decoder part of our system rather that the encoders one. We will show in practice how this can be exploited to fine tune the decoder part to deal with complex localization scenarios in part~\ref{subsec:results}.

%{
\renewcommand{\arraystretch}{1.5}
\begin{table}
	\caption{\label{tab/data}Required data during training}
	\footnotesize \center
	\begin{tabular}{c | c | c}
	\multicolumn{2}{c|}{\textbf{Training method}} & \textbf{Required data} \\
	\hline
	\multirow{3}{*}{Hallucination} & \textit{Image encoder}	& Triplet of images\\
	\cline{2-3}
								   & \textit{Depth encoder}	& Triplet of depth maps\\
	\cline{2-3}
							       & \textit{Hall encoder}	& Triplet of aligned images+depth\\
	\hline
	\multirow{2}{*}{Our} & \textit{Encoders} & Triplet of images\\
	\cline{2-3}
						 & \textit{Decoder} & Aligned images+depth\\
	\end{tabular}
\end{table}
}

\subsection{Hallucination network for image description}
\label{subsec:hall}
We compare our method with the state of the art system that is capable of training a system with side information, named hallucination network~\cite{Hoffman2016}. Hallucination network is originally designed for object detection and classification in an image. We adapt the work of~\cite{Hoffman2016} to create an image descriptor system benefiting for depth map side modality during training. The system is presented in figure~\ref{fig:hall_method}. The main difference with our proposal is that the side information (\textit{i.e.} the depth map) is reconstructed at a deep feature level rather than at the raw data level. We refer readers to~\cite{Hoffman2016} for more information about the hallucination network.

Advantage of hallucination network over our proposal is that it does not required a decoder network, resulting on a architecture lighter than ours. However, it needs a pre-training step, where image encoder and depth map encoder are trained separately from each other before a final optimization step with the hallucination part of the system. We do not need such initialization for training our system. The training procedure of the hallucination network is more complicated and required more complex data that the proposed method. Indeed in order to train the hallucination network for the task of image description we need to gather triplets of \textit{\{image, depth map\}} peer. The triplet constitution required to know the absolute position of the data~\cite{Arandjelovic2017,Liu2018}, or to use costly algorithms like Structure from Motion (SfM)~\cite{Godard2017,Radenovic2017,Kim2017a}. 

Benefit of our method over the hallucination approach is that we have tow unrelated objectives during training: learning a efficient image representation for localization and learning how to reconstruct scene geometry from an image. That means we can train different part of our system separately, with different source of data. Especially, we can improve the scene geometry reconstruction task with unlocalized \textit{\{image, depth map\}} peers. These weakly annotated data are easier to gather than triplet as we only need calibrated system capable of sensing radiometric and geometric modalities at the same time. We will show in practice how this can be exploited to fine tune the decoder part to deal with complex localization scenarios in part~\ref{subsec:results}.


\section{Experiments}
\label{sec:experiments}

\begin{figure}
	\center
	\includegraphics[width=\linewidth]{vect/dataset_ex}
	\caption{\label{fig:dataset} \textbf{Testing sequence:} we compare our new localization method on four challenging sequences featuring radical visual changes.}
\end{figure}

\subsection{Dataset}
\label{subsec:dataset}
	We use the \textit{Oxford Robotcar} public dataset~\cite{Maddern2016} because it presents, in addition to images of the city, depth measurement from lidar. The learned descriptors have as main modality image and as side information depth from a laser scan. This dataset also has a temporal redundancy: 100 repetitions of a consistent route captured over a period of one year. So we can exploit data about the same place at different periods of time, which is essential to train an image descriptor based on CNN.
	
\noindent\textbf{Training dataset.}
	We select three separated areas as training, validation and testing zones. Three runs on the delimited path to create our training set, acquired at dates: \textit{05-19-2015, 08-28-2015} and \textit{11-10-2015}. We create 400 triplets from these four runs for networks training.	
	
\noindent\textbf{Testing datasets.} We propose four testing scenarios featuring challenging conditions for image based localization:
	\paragraph{Sunny/Overcast localization:} queries have been acquired during a sunny day, whereas the reference data have been acquired one day before where the weather was overcast (query images: 261 / ref. images: 1688). 
	\paragraph{Long-term localization:} queries have been acquired 7 mouths before the reference images under similar weather conditions.(query images: 292 / ref. images: 1080).
	\paragraph{Winter/Summer localization:} queries have been acquired during a snowy day (query images: 112 / ref. images: 1688).
	\paragraph{Night/Day localization:} queries have been acquired at night, resulting in radical visual changes compare to the reference images (query images: 156 / ref. images: 1688).

Query examples are presented in figure~\ref{fig:dataset}.
	
\noindent\textbf{Evaluation metric.} For a given query, reference images are ranked according the cosine similarity score computed over their descriptors. To evaluate the localization performances, we consider two evaluation metrics:
	\setcounter{paragraph}{0}
	\paragraph{Recall @N:} we plot the recall curve regarding the number $N$ of returned candidates, {\it i.e.} a query is considered well localized if one of the top $N$ retrieved images lies inside the $25m$ radius of the ground truth query position.
	\paragraph{Top-1 recall @D} We compute the distance between the top ranked returned database image position and the query ground truth position, and report the percentage of queries located under a threshold D (from 15 to 50 meters), like in~\cite{Zamir2014}. This metric is complementary to \textit{recall @N} because it evaluates the absolute precision of the method.
	
\noindent\textbf{Depth map pre-processing.} Depth modality from \textit{Oxford Robotcar} dataset is extracted from the lidar point cloud. When re-projected in the image frame coordinate, it produces a sparse depth map. Deep convolutional neural networks require dense data as input, so we pre-process these sparse modality maps in order to make them dense. We employ an inpainting algorithm from~\cite{Bevilacqua2017}, that minimizes a global energy cost related to image, depth and reflectance points cloud. 

\subsection{Implementation details}
\label{subsec:implementation}

\noindent\textbf{Architectures and pooling methods.}


\noindent\textbf{Truncated Resnet.}

\subsection{Results}
\label{subsec:results}

Short-terme localisation

Long-terme localisation

Night to day localisation

Winter to summer localisation

\section{Conclusion}
\label{sec:conclusion}

Good results for IBL especially under challenging condition (snow and night)

Well suited for place recognition 

Other modality

Generalizing task \cite{Kendall2017}.
\section*{Acknowledgments}

We would like to acknowledge the French ANR project pLaTINUM (ANR-15-CE23-0010) for its financial support. We also gratefully acknowledge the support of NVIDIA Corporation with the donation of the Tesla K40c GPU used for this research.


\bibliographystyle{IEEEtran}
\bibliography{bib/mendeley,bib/other}

\end{document}
