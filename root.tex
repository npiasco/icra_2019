%%%%%%%%%%%%%%%%%%%%%%%%%%%%%%%%%%%%%%%%%%%%%%%%%%%%%%%%%%%%%%%%%%%%%%%%%%%%%%%%
%2345678901234567890123456789012345678901234567890123456789012345678901234567890
%        1         2         3         4         5         6         7         8

\documentclass[letterpaper, 10 pt, conference]{ieeeconf}  % Comment this line out if you need a4paper

%\documentclass[a4paper, 10pt, conference]{ieeeconf}      % Use this line for a4 paper

\IEEEoverridecommandlockouts                              % This command is only needed if 
                                                          % you want to use the \thanks command

\overrideIEEEmargins                                      % Needed to meet printer requirements.

%In case you encounter the following error:
%Error 1010 The PDF file may be corrupt (unable to open PDF file) OR
%Error 1000 An error occurred while parsing a contents stream. Unable to analyze the PDF file.
%This is a known problem with pdfLaTeX conversion filter. The file cannot be opened with acrobat reader
%Please use one of the alternatives below to circumvent this error by uncommenting one or the other
%\pdfobjcompresslevel=0
%\pdfminorversion=4

% See the \addtolength command later in the file to balance the column lengths
% on the last page of the document

% The following packages can be found on http:\\www.ctan.org
\usepackage{graphicx} % for pdf, bitmapped graphics files
%\usepackage{epsfig} % for postscript graphics files
%\usepackage{mathptmx} % assumes new font selection scheme installed
%\usepackage{times} % assumes new font selection scheme installed
\usepackage{amsmath} % assumes amsmath package installed
\usepackage{amssymb}  % assumes amsmath package installed
\usepackage{tabularx}
\usepackage{multirow}
\usepackage{tabularx}
\usepackage{xcolor}

\title{\LARGE \bf
Learning Scene Geometry for \\
Visual Localization in Challenging Condition.
}

\author{Nathan Piasco${}^{1,2}$, D\'esir\'e Sidib\'e${}^1$, Val\'erie Gouet-Brunet${}^2$ and C\'edric Demonceaux${}^1$% <-this % stops a space
\thanks{${}^1$ Le2i, ERL CNRS VIBOT 6000,  Universit\'e  Bourgogne Franche-Comt\'e}% <-this % stops a space
\thanks{${}^2$ Univ. Paris-Est, LaSTIG MATIS, IGN, ENSG, F-94160 Saint-Mand\'e, France}%
}

\begin{document}

\newcommand{\norm}[1]{\left\lVert#1\right\rVert}
\newcommand\numberthis{\addtocounter{equation}{1}\tag{\theequation}}

\maketitle
\thispagestyle{empty}
\pagestyle{empty}


%%%%%%%%%%%%%%%%%%%%%%%%%%%%%%%%%%%%%%%%%%%%%%%%%%%%%%%%%%%%%%%%%%%%%%%%%%%%%%%%
\begin{abstract}
We propose a new approach for outdoor large scale image based localization that can deal with challenging scenarios like cross-season, cross-weather, day/night and long-term localization. The key component of our method is a new learned global image descriptor, that can effectively benefit from scene geometry information during training. At test time, our system is capable of inferring the depth map related to the query image and use it to increase localization accuracy.

We are able to increase recall@1 performances by 2.15\% on cross-weather and long-term localization scenario and by 4.24\% points on a challenging winter/summer localization sequence versus state of the art methods. Our method can also use weakly annotated data to localize night images across a reference dataset of daytime images.
\end{abstract}

\section{Introduction}
\label{sec:intro}

Visual-Based Localization (VBL) is a central topic in robotics and computer vision applications~\cite{Piasco2017}. It consists in retrieving the location of a visual query according to a known absolute reference. VBL is used in many applications such as autonomous driving, augmented reality, urban navigation or automatic map updating. VBL can be seen as an image retrieval problem (or even place recognition task) where an input image is compared to a reference base of localized images. Numerous works have introduced image descriptors well suited for localization~\cite{Arandjelovic2017,Kim2017a,Gordo2017,Radenovic2017,Liu2018}. 

The main challenge of image based localization is to match images under changing conditions: cross-season image matching~\cite{Naseer2017a}, long-term localization~\cite{Toft2018}, day to night place recognition~\cite{Torii2015}, etc. Recent approaches use an extra modality paired with the image in order to address challenging localization scenarios. Scene geometry is often used~\cite{Cavallari?,Schonberger2018}, as well as semantic interpretation~\cite{Ardeshir2014,Christie2016,Naseer2017a}. However geometric or semantic informations are not always available, especially in robotic applications when the sensor load on the robot is limited.

In this paper we propose a method that learn scene geometry related to an image, in order to deal with challenging outdoor large-scale image based localization scenario. We use paired images and depth maps to train a efficient descriptor from image localization. The geometric information provided during the training step make our new descriptor robust to visual changes that occur between images taken at different time. Once trained, our system is used on only images to construct a powerful descriptor. This system design is also known as side information learning~\cite{Hoffman2016}. Our method is especially well suited for robot long-term localization when perceptive sensor on the robot is limited to a camera~\cite{Middelberg2014} but we have an off-line access to the scene geometry~\cite{Paparoditis2012,Maddern2016,Wang2016}. 

The paper is organized as follows. We firstly review in section~\ref{sec:related_work} recent work related to our method including:~state of the art image descriptors for large scale outdoor localization, method for localization in changing environment and side information learning approaches. In section~\ref{sec:method}, we describe in detail our new image descriptor trained with side depth information. We illustrate the effectiveness of the proposed method on four challenging datasets in section~\ref{sec:experiments}. Section~\ref{sec:conclusion} finally concludes the paper.
\section{Related Work}
\label{sec:related_work}

\subsection{Visual-based localization}\label{se:soa_VBL}
Visual-based localization area includes image retrieval-based methods that return the position and orientation of the closest example in the reference data pool. It consists in designing image descriptor dedicated to place recognition. Classical algorithms involve sparse features extraction and description combined with aggregation in order to produce global descriptors. Efficient nearest neighbor search is then used to recover the closest image (or the closest images) in the geo-referenced database~\cite{Arandjelovic2014,Torii2015}. Latest work introduce convolutional neural network for global features extraction~\cite{Arandjelovic2017,Kim2017a,Gordo2017,Radenovic2017,Sunderhauf2015a}.
%\vspace{5pt}
%\noindent Hence both branch of VBL methods can benefit from deep learning, we found natural to apply our learning with side modality approach to the localization task.

\subsection{Learning with side information}\label{se:soa_side}
\paragraph{Learning with side modality} Recent work on side modality learning have been applied for image classification~\cite{Zhang2014}. In this work, authors exploit extra-depth information or multi-spectral channels paired with images. Gupta et al.~\cite{Gupta2016a} use RGB images to train an effective classifier for depth modality with limited number of depth examples. The \textit{supervision transfer} is operated through deep features. The same research team~\cite{Hoffman2016a} explores the benefit of adding an auxiliary modality at test time although no example have been provided for this modality during the training process. The target task was object detection and categorization and the modalities involved were image and depth. The closest work to ours has been presented in~\cite{Hoffman2016}. Authors train a deep neural network to hallucinate features from a modality only present during the training process. In a same manner, Xu et al.~\cite{xu2017learning} use recreated thermal images to improve pedestrian detection.

\subsection{Modality transfer through CNN}\label{se:soa_CNN}
Deep neural network have recently proved high capability in modality transfer~\cite{Kuznietsov2017,Godard2017,xu2017learning}. In particular, fully convolutional architecture seems to be well suited for that task~\cite{Badrinarayanan2017}. For instance, estimating depth map from single RGB data is a challenging problem where fully convolutional network outperforms other methods~\cite{Kuznietsov2017,Godard2017}. Thermal map~\cite{xu2017learning} can also be learned from RGB images through deep neural net. Such network can also model dynamic changes across a single modality, like illumination variation~\cite{Yin2017} or inter-seasonal changes~\cite{zhu2017unpaired}. Based on these recent works, we note that fully convolutional architecture should be well suited for learning various modalities appearances.
\section{Method}
\label{sec:method}

\subsection{Overview}
\label{subsec:overview}

\subsection{Weakly supervised system augmentation}
\label{subsec:sys_augmentation}
\section{Experiments}
\label{sec:experiments}

\begin{figure}
	\center
	\includegraphics[width=\linewidth]{vect/dataset_ex}
	\caption{\label{fig:dataset} \textbf{Testing sequence:} we compare our new localization method on four challenging sequences featuring radical visual changes.}
\end{figure}

\subsection{Dataset}
\label{subsec:dataset}
	We use the \textit{Oxford Robotcar} public dataset~\cite{Maddern2016} because it presents, in addition to images of the city, depth measurement from lidar. The learned descriptors have as main modality image and as side information depth from a laser scan. This dataset also has a temporal redundancy: 100 repetitions of a consistent route captured over a period of one year. So we can exploit data about the same place at different periods of time, which is essential to train an image descriptor based on CNN.
	
\noindent\textbf{Training dataset.}
	We select three separated areas as training, validation and testing zones. Three runs on the delimited path to create our training set, acquired at dates: \textit{05-19-2015, 08-28-2015} and \textit{11-10-2015}. We create 400 triplets from these four runs for networks training.	
	
\noindent\textbf{Testing datasets.} We propose four testing scenarios featuring challenging conditions for image based localization:
	\paragraph{Sunny/Overcast localization:} queries have been acquired during a sunny day, whereas the reference data have been acquired one day before where the weather was overcast (query images: 261 / ref. images: 1688). 
	\paragraph{Long-term localization:} queries have been acquired 7 mouths before the reference images under similar weather conditions.(query images: 292 / ref. images: 1080).
	\paragraph{Winter/Summer localization:} queries have been acquired during a snowy day (query images: 112 / ref. images: 1688).
	\paragraph{Night/Day localization:} queries have been acquired at night, resulting in radical visual changes compare to the reference images (query images: 156 / ref. images: 1688).

Query examples are presented in figure~\ref{fig:dataset}.
	
\noindent\textbf{Evaluation metric.} For a given query, reference images are ranked according the cosine similarity score computed over their descriptors. To evaluate the localization performances, we consider two evaluation metrics:
	\setcounter{paragraph}{0}
	\paragraph{Recall @N:} we plot the recall curve regarding the number $N$ of returned candidates, {\it i.e.} a query is considered well localized if one of the top $N$ retrieved images lies inside the $25m$ radius of the ground truth query position.
	\paragraph{Top-1 recall @D} We compute the distance between the top ranked returned database image position and the query ground truth position, and report the percentage of queries located under a threshold D (from 15 to 50 meters), like in~\cite{Zamir2014}. This metric is complementary to \textit{recall @N} because it evaluates the absolute precision of the method.
	
\noindent\textbf{Depth map pre-processing.} Depth modality from \textit{Oxford Robotcar} dataset is extracted from the lidar point cloud. When re-projected in the image frame coordinate, it produces a sparse depth map. Deep convolutional neural networks require dense data as input, so we pre-process these sparse modality maps in order to make them dense. We employ an inpainting algorithm from~\cite{Bevilacqua2017}, that minimizes a global energy cost related to image, depth and reflectance points cloud. 

\subsection{Implementation details}
\label{subsec:implementation}

\noindent\textbf{Architectures and pooling methods.}


\noindent\textbf{Truncated Resnet.}

\subsection{Results}
\label{subsec:results}

Short-terme localisation

Long-terme localisation

Night to day localisation

Winter to summer localisation

\section{Conclusion}
\label{sec:conclusion}

We have introduced a new competitive global image descriptor designed for image-based localization under challenging conditions. Our descriptor handle visual changes between images by learning the geometry of the scene. Strength of our method remains in the fact that it needs geometric information only during the learning procedure. Our trained descriptor is then used on image only. We show that our proposal is much more efficient than state-of-the-art localization methods~\cite{Arandjelovic2017, Radenovic2017} even on methods based on side information learning~\cite{Hoffman2016}. Our descriptor performs especially well for challenging cross-season localization scenario, therefore it can be used to solve long-term place recognition problem. We additionally obtain encouraging results for night to day image retrieval.

In a future work we will investigate the use of other modalities as side information sources, like the reflectance factor provided by lidars. We also want to study the generalization capability of our system, by considering a different image-based localization task like direct pose regression~\cite{Brachmann2018}.
\section*{Acknowledgments}

We would like to acknowledge the French ANR project pLaTINUM (ANR-15-CE23-0010) for its financial support. We also gratefully acknowledge the support of NVIDIA Corporation with the donation of the Titan Xp GPU used for this research.


\bibliographystyle{IEEEtran}
\bibliography{bib/mendeley,bib/other}

\end{document}
