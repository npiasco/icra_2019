\section{Introduction}
\label{sec:intro}

Visual-Based Localization (VBL) is a central topic in robotics and computer vision applications~\cite{Piasco2017}. It consists in retrieving the location of a visual query according to a known absolute reference. VBL is used in many applications such as autonomous driving, augmented reality, urban navigation or automatic map updating. VBL can be seen as an image retrieval problem (or even place recognition task) where an input image is compared to a reference base of localized images. Numerous works have introduced image descriptors well suited for localization~\cite{Arandjelovic2017,Kim2017a,Gordo2017,Radenovic2017,Liu2018}. 

The main challenge of image based localization is to match images under changing conditions: cross-season image matching~\cite{Naseer2017a}, long-term localization~\cite{Toft2018}, day to night place recognition~\cite{Torii2015}, etc. Recent approaches use an extra modality paired with the image in order to address challenging localization scenarios. Scene geometry is often used~\cite{Cavallari?,Schonberger2018}, as well as semantic interpretation~\cite{Ardeshir2014,Christie2016,Naseer2017a}. However geometric or semantic informations are not always available, especially in robotic applications when the sensor load on the robot is limited.

In this paper we propose a method that learn scene geometry related to an image, in order to deal with challenging outdoor large-scale image based localization scenario. We use paired images and depth maps to train a efficient descriptor from image localization. The geometric information provided during the training step make our new descriptor robust to visual changes that occur between images taken at different time. Once trained, our system is used on only images to construct a powerful descriptor. This system design is also known as side information learning~\cite{Hoffman2016}. Our method is especially well suited for robot long-term localization when perceptive sensor on the robot is limited to a camera~\cite{Middelberg2014} but we have an off-line access to the scene geometry~\cite{Paparoditis2012,Maddern2016,Wang2016}. 

The paper is organized as follows. We firstly review in section~\ref{sec:related_work} recent work related to our method including:~state of the art image descriptors for large scale outdoor localization, method for localization in changing environment and side information learning approaches. In section~\ref{sec:method}, we describe in detail our new image descriptor trained with side depth information. We illustrate the effectiveness of the proposed method on four challenging datasets in section~\ref{sec:experiments}. Section~\ref{sec:conclusion} finally concludes the paper.