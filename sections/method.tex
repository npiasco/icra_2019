\section{Method}
\label{sec:method}

\begin{figure}
	\center
	\includegraphics[width=\linewidth]{vect/our_method}
	\caption{\label{fig:our_method} \textbf{Image descriptors training with auxiliary depth data:} two encoders are used for extracting deep features map from the main image modality and the auxiliary reconstructed depth map (inferred from our deep decoder). These features are used to create intermediate descriptors that are finally concatenated in one final image descriptor.}
\end{figure}

\begin{figure}
	\center
	\includegraphics[width=\linewidth]{vect/hall_method}
	\caption{\label{fig:hall_method} \textbf{Hallucination network for image descriptors learning:} we train an hallucination network, inspired from~\cite{Hoffman2016}, for the task of global image description. Unlike the proposed method (see figure~\ref{fig:our_method}), hallucination network reproduces feature maps that would have been obtained by a network trained with depth map rather than the deep map itself.}
\end{figure}

\subsection{Overview}
\label{subsec:overview}
We design a new global image description for the task of image based localization. We first extract dense feature maps from an input image with a convolutional neural network encoder ($E_I$). This feature maps is subsequently used to build a compact representation of the scene ($d_I$). State of the art features aggregation methods can be used to construct the image descriptor, such as MAC~\cite{Radenovic2017} or NetVALD~\cite{Arandjelovic2017}. We enhance this standard image descriptor with side depth map information that is only available during the training process. To do so, a deep fully convolutional neural network decoder ($D_G$) is used to reconstruct the corresponding depth map according to the input image. The reconstructed depth is then used to extract a global depth map descriptor. We follow the same procedure used before: we extract deep feature maps with an encoder ($E_D$) before building the descriptor ($d_D$). Finally, the image descriptor and the depth map descriptor are concatenated into a single descriptor. Figure~\ref{fig:our_method} summarizes the whole process of our new method.

\subsection{Training routine}
\label{subsec:training}
Trainable parameters are $\theta_{I}$ the weights of encoder and descriptor $\{E_I, d_I\}$, $\theta_{D}$ the weights of the encoder and descriptor $\{E_D, d_D\}$ and $\theta_{G}$ the weights of the decoder $D_G$ used for depth map generation. 

For training our system, we follow standard procedure of descriptor learning based on triplet margin losses. A triplet $\{q_{im}, q_{im}^+, q_{im}^-\}$ is composed of an anchor image $q_{im}$, a positive example $q_{im}^+$ representing the same scene as the anchor and an unrelated negative example $q_{im}^-$.
The first triplet loss acting on $\{E_I, d_I\}$ is:
\begin{multline}
	\label{eq:triplet_loss}
	L_{\theta_{I}} = max\left(0, \lambda + \norm{f_{\theta_{I}}(q_{im}) - f_{\theta_{I}}(q_{im}^+)}_2  \right. \\	
	\left. - \norm{f_{\theta_{I}}(q_{im}) - f_{\theta_{I}}(q_{im}^-)}_2 \right),
\end{multline}
where $f_{\theta_{I}}(x_{im})$ is the global descriptor of image $x_{im}$ and $\lambda$ an hyper-parameter controlling the margin between positive and negative examples. $f_{\theta_{I}}$ can be written as:
\begin{equation}
	\label{eq:desc_details}
	f_{\theta_{I}}(x_{im}) = d_I(E_I(x_{im})),
\end{equation}
where $E_I(x_{im})$ represents the deep feature maps extracted by the decoder and $d_I$ the function used to build the final descriptor from the feature.

We train the depth map encoder and descriptor $\{E_D, d_D\}$ in a same manner, equation~\ref{eq:triplet_loss} becoming:
\begin{multline}
	\label{eq:depth_triplet_loss}
	L_{\theta_{D}}  = max\left(0, \lambda + \norm{f_{\theta_{D}}(\hat{q}_{depth}) - f_{\theta_{D}}(\hat{q}_{depth}^+))}_2  \right. \\	
	\left. - \norm{f_{\theta_{D}}(\hat{q}_{depth}) - f_{\theta_{D}}(\hat{q}_{depth}^-)}_2 \right),
\end{multline}
where $f_{\theta_{D}}(x_{depth})$ is the global descriptor of depth map $x_{depth}$ and $\hat{x}_{depth}$ is the reconstructed depth map of image $x_{im}$ by the decoder $D_G$:
\begin{equation}
	\label{eq:generator}
	\hat{x}_{depth} = D_G(E_I(x_{im})).
\end{equation}
Decoder $D_G$ use the deep representation of image $x_{im}$ computed by encoder $E_I$ in order to reconstruct the scene geometry. Notice that even if the encoder $E_I$ is not especially trained for depth map reconstruction, its intern representation is rich enough to be used by the decoder $D_G$ for the task of depth map inference. We choose to use the features already computed by the first encoder $E_I$ instead of introducing another encoder for saving computational resources.

The final image descriptor is trained with the following loss:
\begin{multline}
	\label{eq:cat_triplet_loss}
	L_{\theta_{I},\theta_{D}} = max\left(0, \lambda + \norm{F_{\theta_{I},\theta_{D}}(q_{im}) - F_{\theta_{I},\theta_{D}}(q_{im}^+)}_2  \right. \\	
	\left. - \norm{F_{\theta_{I},\theta_{D}}(q_{im}) - F_{\theta_{I},\theta_{D}}(q_{im}^-)}_2 \right),
\end{multline}
where $F_{\theta_{I},\theta_{D}}(x_{im})$ denotes the concatenation of image descriptor and depth map descriptor:
\begin{equation}
	\label{eq:cat_function}
	F_{\theta_{I},\theta_{D}}(x_{im}) = \left[ f_{\theta_{I}}(x_{im}), f_{\theta_{D}}(\hat{x}_{depth}) \right].
\end{equation}

In order to train the depth map generator, we use a simple $L_1$ loss function:
\begin{align*}
	L_{\theta_{G}} &= \norm{x_{depth} - \hat{x}_{depth}}_{1}\\
				&= \norm{x_{depth} - D_G(E_I(x_{im}))}_{1}. \numberthis \label{eq:l1_loss}
\end{align*}

The whole system is trained according to the following constraints:
\begin{align}
	\left( \theta_{I}, \theta_{D} \right) & := arg\,\underset{\theta_{I}, \theta_{D}}{min} \left[ L_{\theta_{I}} + L_{\theta_{D}} + L_{\theta_{I},\theta_{D}} \right], \label{eq:sys_optimization_1} \\ 	
	\left( \theta_{G} \right) & := arg\,\underset{\theta_{G}}{min} \left[ L_{\theta_{G}} \right]. 	\label{eq:sys_optimization_2}
\end{align}

In order to train our system, we use two different optimizers: one updating $\theta_{I}$ and $\theta_{D}$ weights regarding constraint~\ref{eq:sys_optimization_1} and the other updating $\theta_{G}$ weights regarding constraint~\ref{eq:sys_optimization_2}. Because decoder $D_G$ is using feature maps computed by encoder $E_I$ (see equation~\ref{eq:generator}), at each optimization step on $\theta_{I}$ we need to update decoder weights $\theta_{G}$ to take in account possible changes in the image features. We finally train our entire system alternating between the optimization of weights $\{\theta_{I}, \theta_{D}\}$ and $\{\theta_G\}$ until convergence.

%\subsection{Fine tuning with weakly annotated data}
%\label{subsec:data}
%As we use triplet margin losses for training encoders weights $(\theta_{I}, \theta_{D})$, we need to gather images triplets. Triplets can be constructed thanks to an absolute position information about images, like GPS-tag~\cite{Arandjelovic2017,Liu2018} or by autonomous computer vision algorithm like Structure from Motion (SfM)~\cite{Godard2017,Radenovic2017,Kim2017a}. In addition, images used for triplets creation must represent the same scenes over different period of time to make the image descriptor robust to visual changes.
%
%In contrast the decoder part of our system $(\theta_{G})$ only needs image depth map aligned pair to be trained. Such data can be easily collected by a mapping mobile system, without using any post-processing. That means we have much more data available for training the decoder part of our system rather that the encoders one. We will show in practice how this can be exploited to fine tune the decoder part to deal with complex localization scenarios in part~\ref{subsec:results}.

%{
\renewcommand{\arraystretch}{1.5}
\begin{table}
	\caption{\label{tab/data}Required data during training}
	\footnotesize \center
	\begin{tabular}{c | c | c}
	\multicolumn{2}{c|}{\textbf{Training method}} & \textbf{Required data} \\
	\hline
	\multirow{3}{*}{Hallucination} & \textit{Image encoder}	& Triplet of images\\
	\cline{2-3}
								   & \textit{Depth encoder}	& Triplet of depth maps\\
	\cline{2-3}
							       & \textit{Hall encoder}	& Triplet of aligned images+depth\\
	\hline
	\multirow{2}{*}{Our} & \textit{Encoders} & Triplet of images\\
	\cline{2-3}
						 & \textit{Decoder} & Aligned images+depth\\
	\end{tabular}
\end{table}
}

\subsection{Hallucination network for image description}
\label{subsec:hall}
We compare our method with the state of the art system that is capable of training a system with side information, named hallucination network~\cite{Hoffman2016}. Hallucination network is originally designed for object detection and classification in an image. We adapt the work of~\cite{Hoffman2016} to create an image descriptor system benefiting for depth map side modality during training. The system is presented in figure~\ref{fig:hall_method}. The main difference with our proposal is that the side information (\textit{i.e.} the depth map) is reconstructed at a deep feature level rather than at the raw data level. We refer readers to~\cite{Hoffman2016} for more information about the hallucination network.

Advantage of hallucination network over our proposal is that it does not required a decoder network, resulting on a architecture lighter than ours. However, it needs a pre-training step, where image encoder and depth map encoder are trained separately from each other before a final optimization step with the hallucination part of the system. We do not need such initialization for training our system. The training procedure of the hallucination network is more complicated and required more complex data that the proposed method. Indeed in order to train the hallucination network for the task of image description we need to gather triplets of \textit{\{image, depth map\}} peer. The triplet constitution required to know the absolute position of the data~\cite{Arandjelovic2017,Liu2018}, or to use costly algorithms like Structure from Motion (SfM)~\cite{Godard2017,Radenovic2017,Kim2017a}. 

Benefit of our method over the hallucination approach is that we have tow unrelated objectives during training: learning a efficient image representation for localization and learning how to reconstruct scene geometry from an image. That means we can train different part of our system separately, with different source of data. Especially, we can improve the scene geometry reconstruction task with unlocalized \textit{\{image, depth map\}} peers. These weakly annotated data are easier to gather than triplet as we only need calibrated system capable of sensing radiometric and geometric modalities at the same time. We will show in practice how this can be exploited to fine tune the decoder part to deal with complex localization scenarios in part~\ref{subsec:results}.

