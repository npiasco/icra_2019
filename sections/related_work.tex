\section{Related Work}
\label{sec:related_work}

\vspace{4pt}\noindent\textbf{Image descriptor for visual localization.} Original global image descriptors are built by combining sparse features with an aggregation method, such as BoW, fisher vector or VLAD. Specific features weighting scheme dedicated to image localisation have been introduced in~\cite{Arandjelovic2014}. Authors of~\cite{Sattler2016} introduce a reranking rountine to improve the localization performances. More recently, \cite{Arandjelovic2017} introduces NetVLAD, a convolutional neural network that is trained to learn a powerful image representation for image localization. NetVLAD use a fully convolutional encoder to extract from an input image a dense feature maps. These maps are than send through a trainable pooling layer that build the final descriptor. Numerous other CNN image descriptors have been proposed in the literature~\cite{Kim2017a,Gordo2017,Radenovic2017,Sunderhauf2015a,Liu2018} and achieve state of the art localization results. Therefore we use CNN image descriptors as base component in our system.

\vspace{4pt}\noindent\textbf{Localization in challenging condition.} Torii

\vspace{4pt}\noindent\textbf{Learning with side information.} Recent work on side modality learning have been applied for image classification~\cite{Zhang2014}. In this work, authors exploit extra-depth information or multi-spectral channels paired with images. Gupta et al.~\cite{Gupta2016a} use RGB images to train an effective classifier for depth modality with limited number of depth examples. The \textit{supervision transfer} is operated through deep features. The same research team~\cite{Hoffman2016a} explores the benefit of adding an auxiliary modality at test time although no example have been provided for this modality during the training process. The target task was object detection and categorization and the modalities involved were image and depth. The closest work to ours has been presented in~\cite{Hoffman2016}. Authors train a deep neural network to hallucinate features from a modality only present during the training process. In a same manner, Xu et al.~\cite{xu2017learning} use recreated thermal images to improve pedestrian detection.
